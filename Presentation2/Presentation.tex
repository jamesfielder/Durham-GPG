\documentclass[xcolor=dvipsnames,usepdftitle=false]{beamer} 
%\usecolortheme[named=BlueViolet]{structure} 
\usecolortheme[RGB={83,13,88}]{structure} \usetheme[height=7mm]{Boadilla} 
\setbeamertemplate{items}[ball] 
\setbeamertemplate{navigation symbols}{}%remove navigation symbols 

\usefonttheme[onlylarge]{structuresmallcapsserif}
\usefonttheme{professionalfonts}

% Normal usepackage includes
\usepackage{graphicx}
\usepackage{color}
\usepackage{amsfonts}
\usepackage{amssymb}
\usepackage{textcomp}
\usepackage{amsthm}
\usepackage{amsopn}
\usepackage{subfigure}
\usepackage{fancybox}

%Set up command for logo in top right corner
\usepackage[absolute,overlay]{textpos}
\setlength{\TPHorizModule}{1mm}
\setlength{\TPVertModule}{1mm}
\newcommand{\MyLogo}{%
	\begin{textblock}{14}(119.0,1.0) %logo position
		\includegraphics[width=0.8cm]{pictures/logo_only.eps}
	\end{textblock}
}

% This puts in a section ``title slide'' when \section command
% is encountered.  Can easily be changed to subsection
\AtBeginSection{
\begin{frame}
		\MyLogo
\begin{center}
		\structure{\Large \insertsection}
	\end{center}
\end{frame}
}

%Use some script greek letters
\renewcommand\epsilon{\varepsilon}
\renewcommand\phi{\varphi}
\renewcommand\theta{\vartheta}

%Macros for standard sets
\newcommand{\newVar}[2]{\newcommand{#1}{\ensuremath{#2}\xspace}}
\newVar\Naturals{\mathbb{N}}
\newVar\Integers{\mathbb{Z}}
\newVar\Rationals{\mathbb{Q}}
\newVar\Reals{\mathbb{R}}
\newVar\Complex{\mathbb{C}}

%++++++++++++++++++++++++++++++++++++++++++++++++++++++++++++++++++++++++++++++++++++++++++++++++++%
%Macros for math functions that take an argument
\newcommand{\ldb}{\text{\textlbrackdbl}}
\newcommand{\rdb}{\text{\textrbrackdbl}}

\newcommand{\norm}[1]{\ensuremath{\lVert#1\rVert}}
\newcommand{\abs}[1]{\ensuremath{\lvert#1\rvert}}
\newcommand{\ceil}[1]{\ensuremath{\lceil#1\rceil}}
\newcommand{\floor}[1]{\ensuremath{\lfloor#1\rfloor}}
\newcommand{\set}[1]{\ensuremath{\{#1\}}}
\newcommand{\angular}[1]{\ensuremath{\langle#1\rangle}}
\newcommand{\paren}[1]{\ensuremath{(#1)}}
\newcommand{\sset}[1]{ \{ #1 \} }
\newcommand{\dset}[2]{ \{ #1 \mid #2 \} }

\newcommand{\Norm}[1]{\ensuremath{\left\lVert#1\right\rVert}}
\newcommand{\Abs}[1]{\ensuremath{\left\lvert#1\right\rvert}}
\newcommand{\Ceil}[1]{\ensuremath{\left\lceil#1\right\rceil}}
\newcommand{\Floor}[1]{\ensuremath{\left\lfloor#1\right\rfloor}}
\newcommand{\Set}[1]{\ensuremath{\left\{#1\right\}}}
\newcommand{\Angular}[1]{\ensuremath{\left\langle#1\right\rangle}}
\newcommand{\Paren}[1]{\ensuremath{\left(#1\right)}}
\newcommand{\Ssett}[1]{ \{ #1 \} }
\newcommand{\Dset}[2]{ \{ #1 \mid #2 \} }

\newcommand{\Average}[1]{\ensuremath{\{ \! \! \{ #1 \} \! \! \}} }
\newcommand{\Jump}[1]{\ensuremath{\ldb #1 \rdb}}
\newcommand{\average}[1]{\ensuremath{\{ \! \! \{ #1 \} \! \! \}} }
\newcommand{\jump}[1]{\ensuremath{\ldb #1 \rdb}}
\newcommand{\diam}[1]{\ensuremath{ \mathrm{diam}(#1) }}

\newcommand\st{\;|\;}

%Document specific commands
\newcommand{\ve}{\varepsilon}
\newcommand{\dR}{\Reals}
\newcommand{\pO}{{\partial \Omega}}
\newcommand{\pOp}{{\partial \Omega^+}}
\newcommand{\pOm}{{\partial \Omega^-}}
\newcommand{\en}[1]{{|\!|\!| #1 |\!|\!|}}
\newcommand{\be}{\begin{equation}}
\newcommand{\ee}{\end{equation}}
\newcommand{\bea}{\begin{eqnarray}}
\newcommand{\eea}{\end{eqnarray}}
\newcommand{\beaa}{\begin{eqnarray*}}
\newcommand{\eeaa}{\end{eqnarray*}}

\newcommand{\T}[0]{{\ensuremath{\mathcal{T}}}}
\newcommand{\Th}[0]{{\ensuremath{\mathcal{T}_h}}}
\newcommand{\Tp}[0]{{\ensuremath{T^+}}}
\newcommand{\Tm}[0]{{\ensuremath{T^-}}}
\newcommand{\LO}[0]{{\ensuremath{L^2 (\Omega)}}}
\newcommand{\LT}[0]{{\ensuremath{L^2 (T)}}}
\newcommand{\LTp}[0]{{\ensuremath{L^2 (\Tp)}}}
\newcommand{\LTm}[0]{{\ensuremath{L^2 (\Tm)}}}
\newcommand{\Le}[0]{{\ensuremath{L^2 (e)}}}
\newcommand{\LeJ}[0]{{\ensuremath{L^2 (J)}}}
\newcommand{\TinT}[0]{{\ensuremath{T \in \Th}}}
\newcommand{\C}[0]{\ensuremath{{\mathcal{C}}}}
\newcommand{\D}[0]{\ensuremath{{\mathcal{D}}}}
\newcommand{\ed}[0]{\ensuremath{{\epsilon \Delta}}}
\newcommand{\edh}[0]{\ensuremath{{\epsilon \Delta,h}}}
\newcommand{\zeh}[0]{{\ensuremath{0,\epsilon,h}}}

\newcommand{\HOneZero}[0]{\ensuremath{H^1_0(\Omega)}}

\newcommand{\Eho}[0]{\ensuremath{{\mathcal{E}_h^o}}}
\newcommand{\Eh}[0]{\ensuremath{{\mathcal{E}_h}}}
\newcommand{\einEh}[0]{{\ensuremath{e \in \Eh}}}
\newcommand{\einEho}[0]{{\ensuremath{e \in \EHo}}}

\newcommand{\pOC}[0]{{\partial \Omega_{\C}}}
\newcommand{\pOD}[0]{{\partial \Omega_{\D}}}
\newcommand{\OC}[0]{{\Omega_{\C}}}
\newcommand{\OD}[0]{{\Omega_{\D}}}
\newcommand{\LOC}[0]{{\ensuremath{L^2 (\OC)}}}
\newcommand{\HOC}[0]{{\ensuremath{H^1 (\OC)}}}
\newcommand{\LOCJ}[0]{{\ensuremath{L^2 ({\Omega_{\C}^J})}}}
\newcommand{\HOCJ}[0]{{\ensuremath{H^1 ({\Omega_{\C}^J})}}}
\newcommand{\HT}[0]{{\ensuremath{H^1 (T)}}}
\newcommand{\Linfty}[0]{\ensuremath{{L^{\infty} (\Omega)}}}
\newcommand{\LinftyC}[0]{\ensuremath{{L^{\infty} (\OC)}}}
\newcommand{\LinftyD}[0]{\ensuremath{{L^{\infty} (\OD)}}}
\newcommand{\LinftyT}[0]{\ensuremath{{L^{\infty} (T)}}}
\newcommand{\LinftyA}[1]{\ensuremath{{L^{\infty}(#1)}}}
\newcommand{\LpOC}[0]{{\ensuremath{L^2 (\pOC)}}}
\newcommand{\GC}[0]{{\ensuremath{\Gamma_\C}}}
\newcommand{\GD}[0]{{\ensuremath{\Gamma_\D}}}

\newcommand{\TC}[0]{{\T_{\C}}}
\newcommand{\TD}[0]{{\T_{\D}}}
\newcommand{\TinTC}[0]{{\ensuremath{T \in \TC}}}
\newcommand{\TinTD}[0]{{\ensuremath{T \in \TD}}}
\newcommand{\pTm}[0]{\ensuremath{{\partial \Tm}}}
\newcommand{\pTp}[0]{\ensuremath{{\partial \Tp}}}

\newcommand{\CG}[0]{\ensuremath{\text{cG}}}
\newcommand{\CDG}[0]{\ensuremath{\text{cdG}}}
\newcommand{\DG}[0]{\ensuremath{\text{dG}}}
\newcommand{\SDG}[0]{{\ensuremath{\text{SdG}}}}
\newcommand{\rhobar}[0]{\ensuremath{\overline{\rho}}}

\newcommand{\vectau}[0]{\ensuremath{\boldsymbol{\tau}}}

\newcommand{\Poly}[0]{\ensuremath{\mathcal{P}}}

\newcommand{\Obar}[0]{\ensuremath{\overline{\Omega}}}

\renewcommand\qedsymbol{\ensuremath{\square}}

\newcommand{\ex}{{\rm e}}
\newcommand{\trans}[1]{{#1}^{\ensuremath{\mathsf{T}}}}

\newcommand{\sqrtdivb}[0]{\ensuremath{\sqrt{-\nabla \cdot b}}}
\newcommand{\half}[0]{\ensuremath{\frac{1}{2}}}
\newcommand{\fhalf}[0]{\ensuremath{{1/2}}}

\newcommand{\calS}[0]{\ensuremath{\mathcal{S}}}
\newcommand{\SZ}[0]{\ensuremath{\mathcal{SZ}}}
\newcommand{\ProjSZ}[0]{\ensuremath{{\Pi_h^\SZ}}}

\newcommand{\Lproj}[0]{{\ensuremath{\Pi_h}}}
\newcommand{\LprojT}[0]{\ensuremath{\Pi_{h,T}}}
\newcommand{\LprojL}[0]{{\ensuremath{\Pi_h^{L^2}}}}
\newcommand{\LprojLT}[0]{{\ensuremath{\Pi_{h,T}^{L^2}}}}
\newcommand{\LprojH}[0]{{\ensuremath{\Pi_h^{H^1}}}}

\newcommand{\bdotnabv}[0]{{\ensuremath{b \cdot \nabla v}}}

\newcommand{\Ltwo}[1]{\ensuremath{{L^2 (#1)}}}
\newcommand{\Spart}[0]{\ensuremath{{\mathcal{S}}}}
\newcommand{\Hnorm}[2]{\ensuremath{{H^{#1}({#2})}}}
\newcommand{\LJ}[0]{\ensuremath{\Ltwo{J}}}
\newcommand{\Wdiv}[1]{\ensuremath{{W^\infty (\mathrm{div},#1)}}}

\newcommand{\lowlight}[1]{\textcolor[gray]{0.8}{#1}}
\newcommand{\piD}[0]{\ensuremath{{\pi,\D}}}
%++++++++++++++++++++++++++++++++++++++++++++++++++++++++++++++++++++++++++++++++++++++++++++++++++%


% title page stuff
\title[Short Title]{Long Title}
\author{Your Name}
\institute[Durham]{}
%\titlegraphic{\includegraphics[height=1.5cm]{pictures/logo_purple.eps}}
\date[29 July 2010]{Joint work with X (from here),\\Y (from here) and Z (from here)}
\institute[Durham]{\includegraphics[height=1.5cm]{pictures/logo_purple.eps}}

\begin{document}

\frame[plain]{\titlepage}

\begin{frame}
	\MyLogo
	\frametitle{Outline}
	\tableofcontents
\end{frame}

%==================================================================================================%
\section{Section 1}

%--------------------------------------------------------------------------------------------------%
\subsection*{Subsection} %For your own organisation (as far as I can tell) %

\begin{frame}
	\MyLogo
	\frametitle{Frame Title}
	\begin{equation} 
		\begin{split}
		-\epsilon \Delta u + b \cdot \nabla u & = f  \quad \mathrm{in}~ \Omega \subset \mathbb{R}^d  \nonumber\\
		u & = 0   \quad \mathrm{on}~ \pO \nonumber
		\end{split}
	\end{equation}
	Look at these equations...

	\begin{theorem}
		A Theorem
		\begin{equation}
			- \half \nabla \cdot b \ge \rho \ge 0. \nonumber
		\end{equation}
	\end{theorem}
	\begin{proof}
		A Proof
	\end{proof}

\end{frame}

%--------------------------------------------------------------------------------------------------%
\subsection*{Another Subsection}

\begin{frame}
	\MyLogo
	\frametitle{Frame Title}
	There are some pictures below...
	\begin{figure}[htb]
		\setcounter{subfigure}{0}
		\centering
		\subfigure[$\epsilon = 0.1$]{\includegraphics[width=0.32\textwidth]{pictures/1d_e=0-1.eps}}
		\subfigure[$\epsilon = 0.01$]{\includegraphics[width=0.32\textwidth]{pictures/1d_e=0-01.eps}}
		\subfigure[$\epsilon = 0.001$]{\includegraphics[width=0.32\textwidth]{pictures/1d_e=0-001.eps}}
	\end{figure}
\end{frame}

\begin{frame}
	\MyLogo
	\frametitle{Another Frame}
	Some equations and pictures...
	\begin{equation} 
		\begin{split}
		-0.01 \Delta u + (-1,0)^\top \cdot \nabla u & = 1  \quad \mathrm{in}~ \Omega \subset \mathbb{R}^d  \nonumber\\
		u & = 0   \quad \mathrm{on}~ \pO \nonumber
		\end{split}
	\end{equation}
	\begin{figure}[htb]
		\setcounter{subfigure}{0}
		\centering
		\subfigure[Standard plot]{\includegraphics[width=0.48\textwidth]{pictures/2d_cG_plot_e=0-01.eps}}
		\subfigure[Temperature map]{\includegraphics[width=0.48\textwidth]{pictures/2d_cG_tempmap_e=0-01.eps}}
	\end{figure}
\end{frame}

%==================================================================================================%
\section{Section 2}

%--------------------------------------------------------------------------------------------------%
\subsection*{Another one}

\begin{frame}
	\MyLogo
	\frametitle{Frame Title}
	A definition...
	\begin{definition}[Something to Define]
		Some text...
		\begin{equation}
			B_{\epsilon}(u_h,v) := \epsilon (\nabla u_h, \nabla v) + (b.\nabla u_h, v) = (f,v) \quad \forall v \in V_h. \nonumber
		\end{equation}
	\end{definition}

\end{frame}

%--------------------------------------------------------------------------------------------------%
\subsection*{Add more as needed}

\begin{frame}
	\MyLogo
	\frametitle{more?}
	
\end{frame}

%--------------------------------------------------------------------------------------------------%

%==================================================================================================%
\section{Section 3}

%--------------------------------------------------------------------------------------------------%
\subsection*{Definitions and derivation}


\begin{frame}
	\MyLogo
	listing things...
	\begin{itemize}
		\item $\Gamma$ the union of boundary faces (those in $\pO$).
		\item For $e \in \Eho$, $T^+$ is the downwind cell, $T^-$ the upwind cell as determined by $b \cdot n$ on the face from each cell, $n$ being the outward pointing normal.
		\item For $e \in \Eho$ the jump $\jump{\cdot}$ and average $\average{\cdot}$ are defined by
			\begin{equation}
				\begin{split}
					\jump{\nu} = \nu^+ n^+ + \nu^- n^-, &\quad \jump{\vectau} = \vectau^+ \cdot n^+ + \vectau^- \cdot n^- \nonumber \\
					\average{\nu} = \half (\nu^+ + \nu^-), &\quad \Average{\vectau} = \frac{1}{2} (\vectau^+ + \vectau^-). \nonumber
				\end{split}
			\end{equation}
		\item On the boundary these become
			\begin{equation}
				\jump{\nu} = \nu n, \quad \average{\nu} = \nu, \quad \average{\vectau} = \tau. \nonumber
			\end{equation}
		\item A very useful identity
			\begin{equation}
				\sum_{\TinT} \int_{\partial T} \nu \vectau \cdot n = \int_{e\in \Eh} \jump{\nu} \cdot \average{\vectau} + \int_{e\in \Eho} \average{\nu} \jump{\vectau} \nonumber
			\end{equation}
	\end{itemize}
\end{frame}


%==================================================================================================%
\section{Section 4}

%--------------------------------------------------------------------------------------------------%

\subsection*{Further work}

\begin{frame}
	\MyLogo
	\frametitle{Future work on...}
	\begin{itemize}
		\item Idea 1.
		\item \textcolor{Gray}{Idea 2.}
		\item \textcolor{Gray}{Idea 3.}
	\end{itemize}
	
	Extra info...
\end{frame}

\subsection*{References}

\begin{frame}
	\MyLogo
	\frametitle{References}
	\tiny{
	\bibliographystyle{apalike}
	\bibliography{../../../Bibliography/full_bibliography.bib}} % Replace with path to your .bib file
\end{frame}



\end{document}