%\documentclass[a4paper]{leaflet}
%%\pdfpagewidth=420mm
%\pdfpageheight=\paperheight
%\usepackage{lipsum}
%%\usepackage{fontspec}
%\begin{document}
%%\title{GPG: A Guide by DUCS \Lclass{leaflet}}

%\title{The document class \Lclass{leaflet}}
%\author{%
 % Rolf Niepraschk\\
 % Walter Schmidt\\
 % Hubert G\"a\ss lein}
%\date{Last updated~\docdate\\printed \today}
%BLAH
%\end{document}

%%%%%%%%%%%%%%%%%%%%%%%%%%%%%%%%%%%%%%%%%%%%%%%%%%%%%%%%%%%%%%%%

%% Durham University Computing Society's Guide to GPG.
%%
%% Template leaflet file: `leaflet-manual.tex', (licensed under LPPL), official LaTeX Leaflet Manual.
%% 
%% 

%%%%%%%%%%%%%%%%%%%%%%%%%%%%%%%%%%%%%%%%%%%%%%%%%%%%%%%%%%%%%%%%
\def\filename{leaflet.tex}
\def\fileversion{v0.1}   % change this when leaflet-manual changed, too.
\def\filedate{2012/05/29}
\def\docdate {May 2012} % change this when leaflet-manual changed, too.
\listfiles
\errorcontextlines=99
\documentclass[
notumble, %% -- turns bottom page over for proofreading
%%nofoldmark,
%%dvipdfm,
%%portrait,
%%titlepage,
%%nocombine,
%%a3paper,
%%debug,
%%nospecialtricks,
%%draft,
]{leaflet}


\renewcommand*\foldmarkrule{.3mm}
\renewcommand*\foldmarklength{5mm}

\usepackage[T1]{fontenc}
\usepackage{textcomp}
\usepackage{mathptmx}
\usepackage{paralist}
\usepackage[scaled=0.9]{helvet}
\usepackage{hyperref}
\makeatletter
\def\ptmTeX{T\kern-.1667em\lower.5ex\hbox{E}\kern-.075emX\@}
\DeclareRobustCommand{\ptmLaTeX}{L\kern-.3em
        {\setbox0\hbox{T}%
         %\vb@xt@ % :-)
         \vbox to\ht0{\hbox{%
                            \csname S@\f@size\endcsname
                            \fontsize\sf@size\z@
                            \math@fontsfalse\selectfont
                            A}%
                      \vss}%
        }%
        \kern-.12em
        \ptmTeX}
\makeatother
\let\TeX=\ptmTeX
\let\LaTeX=\ptmLaTeX
\usepackage{shortvrb}
\MakeShortVerb{\|}
\usepackage{url}
\usepackage{graphicx}
\usepackage[dvipsnames,usenames]{color}
\definecolor{LIGHTGRAY}{gray}{.9}

%%%%\renewcommand{\descfont}{\normalfont}
%\newcommand\Lpack[1]{\textsf{#1}}
%\newcommand\Lclass[1]{\textsf{#1}}
%\newcommand\Lopt[1]{\texttt{#1}}
%\newcommand\Lprog[1]{\textit{#1}}

%\newcommand*\defaultmarker{\textsuperscript\textasteriskcentered}

\title{\textit{GNU Privacy Guard}: \\A CompSoc guide to daily use of email encryption\\\vskip1.5em Windows Edition}
\author{%
  Martin Dehnel\\
  James Fielder\\ 
  (Durham University)
  }
\date{~\docdate}% \vskip11em}

%\CutLine*{1}% Dotted line without scissors
%\CutLine{6}%  Dotted line with scissors

%\AddToBackground{5}{%  Background of a small page
%  \put(0,0){\textcolor{Cerulean}{\rule{\paperwidth}{\paperheight}}}}

%\AddToBackground*{2}{% Background of a large page
%  \put(\LenToUnit{.5\paperwidth},\LenToUnit{.5\paperheight}){%
%    \makebox(0,0)[c]{%
%      \resizebox{.9\paperwidth}{!}{\rotatebox{35.26}{%
%        \textsf{\textbf{\textcolor{LIGHTGRAY}{BACKGROUND}}}}}}}}

\begin{document}
\maketitle
\includegraphics[width=0.9\paperwidth]{images/logo.png}
%\includegraphics{images/gpg-logo.png}
%\includegraphics[scale=1]{images/gpg-logo.png}
\thispagestyle{empty}

%%\LARGE

%%\tableofcontents

\section{What is GPG?}

\textbf{GPG}, or GNU Privacy Guard is, in a nutshell, \textit{a free and easy way to send and receive emails completely securely}. The way most emails are currently sent is completely insecure, and is directly analogous to sending all of your mail on a postcard, available for any postman or eavesdropper along the way to read: we don't think this is good enough, and want to encourage more people to think and care about their privacy online.

\textbf{GPG} allows you to send \textbf{completely secure} emails to anyone else with an email address and a key: no special email provider is needed.

We think that using GPG is something which everyone should consider, simply because it does away with many issues which are absolutely inherent to email:

\begin{itemize}
	\item It makes it much much more difficult to forge emails between people who are GPG enabled. With signing you can prove with cryptographic techniques that you sent the email, and not someone else. Imagine if someone emailed your employer and told them you were quiting. If you signed all your email, they might doubt that it is a valid email.
	\item If you use Gmail or other such services it allows you to negate having to trust them not to give your email to third parties, or read them themselves. While they can still read signed emails, your encrypted emails would be unreadable for anyone but you and the recipient. Indeed, the FBI can get a copy of your email without even needing a warrant if you are not a US citizen from Google.  
\end{itemize}

\section{How does it work?}
To make sure that only the intended recipient can read the email you've sent them, you need to encrypt the message. This means that to anyone without the right password (or `key') the message will look like random gibberish. \\As you may or may not have ever met the person you want to email, you probably won't have a way of agreeing a password without knowing for certain that no-one can intercept it (meaning they could read your messages too). Instead, as an analogy, you create an \textit{electronic padlock and key}, or \textbf{Public / Private Key Pair}. The Public Key is the padlock, and the Private Key is the key you use to unlock it. You mustn't send anyone a copy of the private key over the internet as someone could copy it, but instead you can send out an unlocked padlock (the public key) to anyone who wants to send you an email; this way they can `lock' (encrypt) the message with the padlock, but no-one, not even the sender can `unlock' (decrypt) it -- only you, the person with the private key can do that.\\ So that anyone can send you an email securely, you distribute your `open padlock' (Public Key) freely using a Key Server, such as \href{http://pgp.mit.edu}{http://pgp.mit.edu}. Anyone can type in your name or email address and find your public key this way, but don't worry, there's no way anyone can work backwards from the Public Key to work out anything about your Private Key.

\section{Setup and usage guide, Windows edition}

This guide describes the use of Thunderbird with Enigmail. There are many other setups for windows available, including one for Outlook. This guide also assumes you have setup Thunderbird to connect with your email account. Instructions for a durham.ac.uk account are \href{http://www.dur.ac.uk/cis/email/exchange/eximap/configuring-clients/}{here}. A more comprehensive guide with screenshots appears on \href{http://durhamgpg.co.uk}{http://durhamgpg.co.uk}

\begin{compactenum}[1.]
    \item Install Thunderbird from the Mozilla website, and Gpg4win from their website. Install the Enigmail extension from the addons page in Thunderbird. This is found by going to \textbf{tools>addons} and then searching and installing Enigmail from the addons page.
    \item Restart Thunderbird to enable Enigmail.
    \item On restarting Thunderbird you will notice a new menu at the top appear called OpenPGP. Open this menu and go to Key Management. In Key management there should be an option called Generate at the top, select this option.
    \item If you have correctly setup your email account an identity should already be selected in the Generate menu. Make sure the identity you wish to generate a key for is selected, and set the options for generation as follows: 
        \begin{itemize}
            \item Key expires in 5 years. This is best as... think of something cool here.
            \item In advanced choose an RSA key with 4096 bits.
            \item Pick a suitably long passphrase (greater than 15 characters) and \textbf{do not forget it}. A passphrase prevents your private key from being used by anyone, even if they get access to your computer. However without the passphrase you will be unable to use the key.
            \item You may wish to add a comment to the key. This can be anything you like, and is not required.
        \end{itemize}
    \item While the key is generating make sure to use your computer for other things, as in order to make sure the key is secure gpg requires a good source of random data.
    \item At the end of the key generation you will be asked if you wish to generate a revocation certificate. This certificate allows you to mark your public key as no longer valid if you lose access to your private key. Choose yes and make sure to keep the file produced safe. While you can generate a revocation certificate while you have your private key, if you loose it you will no longer be able to.
    \item All that remains to do now is to set Enigmail to automatically sign all emails and to use PGP/MIME when doing so. Go to \textbf{options>accounts} and select ``OpenPGP Security'' from the options at the side. Enable OpenPGP support if it is not already enabled, and select to sign both non-encrypted and encrypted messages by default and to use PGP/MIME by default.
    \item At this point you will be ready to try sending an encrypted and signed email to someone. There is an address setup just for this purpose at 'test@durhamgpg.co.uk'. However, first you will have to retrieve the public key for this address. This can be done from within Thunderbird by going to the Key Management menu, looking under Keyserver and searching for a key with ID 0xB76176FE which is the key for test@durhamgpg.co.uk. If you are on the Durham network this might not work, as it requires odd ports (port 11371) to work. If this is the case make sure your browser is configured as per \href{http://durhamgpg.co.uk/ffproxy.png}{here} and search for the key at \href{http://pgp.mit.edu}{http://pgp.mit.edu}. The key for test@durhamgpg.co.uk for example is \href{http://pgp.mit.edu:11371/pks/lookup?op=get&search=0x773B90C7B76176FE}{here}. Copy and paste this into Thunderbird if neccessary using \textbf{Edit>Import Keys from Clipboard} in Key Management.
    \item It would also be a good idea for you to put your public key onto a keyserver so that others can find your key and email you using it. If you click "Display all Keys by Default" in the Key Management window you will see your generated key. If you right click on your key (It will be in bold) and select "Copy Public Keys to Clipboard" and then paste it into the upload keys box on the front of \href{http://pgp.mit.edu}{http://pgp.mit.edu}.
\end{compactenum}

\section{General Advice}

\begin{itemize}
    \item One of the best features of setting up OpenGPG for an email identity is the ability is cryptographically prove that it was you who sent the email. Thus we suggest that you sign all emails you send, so that if someone does try to impersonate you over email, you can show that you sign all your emails and therefore since the imposter does not have your private key and cannot sign as you, you could not have possibly sent the email.
    \item Make sure to keep good backups of your keys and revocation certificates. You cannot regenerate a private key from a public one, for obvious reasons.
    \item In 5 years time, when your key expires, it may be more prudent to replace it with an ECDSA key. Although check if this is still good advice at the time.
\end{itemize}

\section{Key Signing and Web of Trust}

While reading this guide, you might have realized a slight flaw in the system: anyone could generate a key pretending to be you, and then people would start using this key as though it is yours once they found it on the database. However, there is a solution to this, which is called key signing. If a key is signed by another persons key it means that that those people have met, exchanged ID to prove who they are, and signed each others keys. This way, you end up with a web of trusted keys between people, and can see if a key is legitimate or not. 

A guide to doing this is beyond the scope of this leaflet (and there isn't really space). More details of how to do this will be included on the website.


\section{Disadvantages}

There are a few disadvantages with using GPG, which we feel we should acknowledge. Firstly, using GPG with webmail is currently a difficult task, although there are projects like \href{http://gpgsend.com/}{http://gpgsend.com/} and \href{http://openpgpjs.org/}{http://openpgpjs.org/} which are attempting to make it easier to use. Remember, without the setup, there is no way you will be able to read encrypted emails in webmail. Reading signed ones will be fine however.

Most people aren't setup to use GPG, and therefore sending them signed messages may solicit questions from them. Feel free to send them towards our guides at \href{http://durhamgpg.co.uk}{http://durhamgpg.co.uk}.

\end{document}
